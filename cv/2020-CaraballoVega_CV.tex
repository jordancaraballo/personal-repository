%%%%%%%%%%%%%%%%%%%%%%%%%%%%%%%%%%%%%%%%%
% Medium Length Professional CV
% LaTeX Template
% Version 2.0 (8/5/13)
%
% This template has been downloaded from:
% http://www.LaTeXTemplates.com
%
% Original author:
% Trey Hunner (http://www.treyhunner.com/)
%
% Important note:
% This template requires the resume.cls file to be in the same directory as the
% .tex file. The resume.cls file provides the resume style used for structuring the
% document.
%
%%%%%%%%%%%%%%%%%%%%%%%%%%%%%%%%%%%%%%%%%

%----------------------------------------------------------------------------------------
%	PACKAGES AND OTHER DOCUMENT CONFIGURATIONS
%----------------------------------------------------------------------------------------

\documentclass{resume} % Use the custom resume.cls style

\usepackage[left=0.75in,top=0.6in,right=0.75in,bottom=0.6in]{geometry} % Document margins

\name{Jordan A. Caraballo-Vega} % Your name
\address{786-368-1596 \\ jordan.a.caraballo.vega@gmail.com} % Your phone number and email
\address{github.com/jordancaraballo \\ linkedin.com/in/jordancaraballovega} % Your phone number and email

\begin{document}

%----------------------------------------------------------------------------------------
%	EDUCATION SECTION
%----------------------------------------------------------------------------------------

\begin{rSection}{Education}

{\bf University of Puerto Rico at Humacao} \hfill {\em 2015-2020} \\ 
Bachelor of Science - Computational Mathematics \\
Major Computer Science, GPA: 3.91/4.0 \smallskip \\
\end{rSection}

%----------------------------------------------------------------------------------------
%	RESEARCH INTERESTS SECTION
%----------------------------------------------------------------------------------------
\begin{rSection}{Research Interests}
\begin{itemize}
\item Intersection between \textbf{remotely sensed data}, \textbf{artificial intelligence} and \textbf{high-performance computing} applications. 
\item Utilization of very-high resolution multi-spectral imagery for \textbf{land use} and \textbf{land cover} change monitoring, \textbf{feature detection} in more than-regional study areas,  and \textbf{environmental impacts} on human health.
\item \textbf{Wildland fire} forecasting, occurrence \textbf{monitoring} and characterization, including its \textbf{environmental} and \textbf{ecological} impacts.
\end{itemize}
\end{rSection}

%----------------------------------------------------------------------------------------
%	WORK EXPERIENCE SECTION
%----------------------------------------------------------------------------------------

\begin{rSection}{Professional Experience}

%------------------------------------------------
\begin{rSubsection}{Science Data Processing Branch, Goddard Space Flight Center}{Fall 2018 - Present}{Computer Engineer AST}{Greenbelt, Maryland}
Actively performed research applied to mid- and very-high resolution multi-spectral remotely sensed data under Dr. Carroll's Innovation Lab Team.  Several areas of previous research experience include land cover and land use change, thaw slump and mangrove instance segmentation, cloud masking of WorldView imagery, crop mapping, and several other Earth surface monitoring and analysis applications.  Developed innovative software and applications for the training and inference of Earth science artificial intelligence models at scale using parallelization and GPU computing techniques.  Aided the Global Modeling and Assimilation Office (GMAO) in the benchmarking and deployment of software.
\end{rSubsection}

%------------------------------------------------

\begin{rSubsection}{Partnership for Research and Education in Materials (PREM), NSF}{2013 - 2020}{Computational Research Assistant}{Humacao, PR}
\item Lead a team of five undergraduate students in the development and modeling of health and biological material science applications. Developed and implemented software to accelerate 3-dimensional molecular dynamics simulations in HPC environments. Assisted and collaborated with ongoing research projects in data mining, clustering, image processing, and machine learning applied to vector and spatially-oriented data. Studied biological and electrical systems applied to physical sensors by means of molecular dynamics simulations and big data analytics.
\end{rSubsection}

\begin{rSubsection}{NASA Center for Climate Simulation, Goddard Space Flight Center}{June 2016 - July 2018}{Computer Science Trainee}{Greenbelt, Maryland}
\item[] Intern: Summer 2016, Fall 2016, Summer 2017, Summer 2018
\item[] Contractor under ADNET LLC: Spring 2017, Fall 2017, Spring 2018
\item Developed high-performing scientific software to support several compute intensive applications via multi-node MPI implementations and DevSecOps initiatives. Provided extensive support in the development and analysis of Web Applications, including Linux and Unix system administrator for security and resource intensive systems, including revision control, networking, cloud environments, and user-facing virtual machines to support the entire Super Computing Center.
\end{rSubsection}
\end{rSection}

%----------------------------------------------------------------------------------------
%	TECHNICAL STRENGTHS SECTION
%----------------------------------------------------------------------------------------

\begin{rSection}{Technical Strengths}

\begin{tabular}{ @{} >{\bfseries}l @{\hspace{6ex}} l }
Remote Sensing Skills & QGIS, ArcGIS, Google Maps API, OpenStreetMap, GDAL, Rasterio\\
Remote Sensing Imagery & Very-high resolution Digital Globe data, MODIS, Landsat, Sentinel-2\\
Programming Languages & Python, C++, Perl, NodeJS, Bash \\
Computing Skills & Data Mining, Image Processing, Computer Vision, GPU, MPI, Xarray\\
Machine/Deep Learning &  TensorFlow, PyTorch, NVIDIA RAPIDS, SkLearn \\
Modeling Skills & NAMD2, LAMMPS, AutoCAD, Spatial and non-spatial modeling \\
DevOps & Containers, DevSecOps, Kubernetes,  Cloud Computing (AWS, GCP) \\
Operating Systems & System Administration of Linux, Unix, WindowsR2, FreeBSD, \\
Website Development & HTML5, CSS, Javascript, Drupal \\
Soft Skills & Advanced Spanish, Advanced English, Effective Conflict Management \\
Security Clearance & Public Trust
\end{tabular}

\end{rSection}
%----------------------------------------------------------------------------------------
%	Publications
%----------------------------------------------------------------------------------------

\begin{rSection}{Selectd Peer Reviewed Conference Publications/Presentations}
\textbf{Abstract} - Caraballo-Vega, J (2021) Towards Scalable \& GPU Accelerated Earth Science Imagery Processing: An AI/ML Case Study,  AGU,  New Orleans, LA, Published Presentation. \\
\textbf{Abstract} - Caraballo-Vega, J (2021) Ensemble Learning Methods \& Deep Learning for the Task of Cloud Detection, AGU, New Orleans, LA, Published Poster. \\
\textbf{Abstract} - Cantu, L; Caraballo-Vega, J (2021) Application of the Data Science Workflow to Molecular Dynamic Simulations, Published Poster. \\
\textbf{Conference Proceeding} - JAC, Mir., F.M. (2014) Molecular dynamics simulation of electrodes for capacitors made with nano-onions, NCUR. \\
\end{rSection}


%----------------------------------------------------------------------------------------
%	Most Recent Presentations
%----------------------------------------------------------------------------------------

\begin{rSection}{Selected Recent Presentations}
Caraballo-Vega, J (2021) An Integrated Approach to Artificial Intelligence: Accelerating Science, Standardizing Operations, PREP-II, NASA Goddard Space Flight Center, Greenbelt, MD. \\
Caraballo-Vega, J (2021) Towards Scalable \& GPU Accelerated Earth Science Imagery Processing: An AI/ML Case Study,  AGU, New Orleans, LA.\\
Caraballo-Vega, J (2021) Ensemble Learning Methods \& Deep Learning for the Task of Cloud Detection, AGU, New Orleans, LA.\\
Caraballo-Vega, J (2021) Thaw Slump Characterization in the Artic: An Instance Segmentation Problem, On the Pathway to a Digital Earth, UMD.\\
Caraballo-Vega, J (2021) Closing the Gap in AI/ML Software Reproducibility by Containerizing Applications: An Earth Science Case Study, On the Pathway to a Digital Earth, UMD.\\
Caraballo-Vega, J (2021) Applicability of Mask R-CNN for Object Classification in Very High-Resolution Satellite Imagery,  3rd NOAA AI Workshop, Virtual. \\
Caraballo-Vega, J (2021) From Computers to Earth Science: Finding My True Passion,  UPRH Physics and Aerospace Careers Panelist, Virtual. \\
Caraballo-Vega, J (2020) Deep Learning Techniques for the Classification of VHR Resolution Satellite Imagery, PREP-I, NASA Goddard Space Flight Center, Greenbelt, MD. \\
\end{rSection}

%----------------------------------------------------------------------------------------
%	TECHNICAL STRENGTHS SECTION
%----------------------------------------------------------------------------------------

\begin{rSection}{Internships}

\begin{tabular}{ @{} >{\bfseries}l @{\hspace{6ex}} l }
June-August 2018 & Summer Internship, NASA Goddard Space Flight Center, Greenbelt, MD. \\
June-August 2017 & Summer Internship, NASA Goddard Space Flight Center, Greenbelt, MD. \\
August-December 2016 & Fall Internship, NASA Goddard Space Flight Center, Greenbelt, MD. \\
June-August 2016 & Summer Internship, NASA Goddard Space Flight Center, Greenbelt, MD. \\
June-August 2014 & REU, University of Pennsylvania, Philadelphia, PA. \\
June-August 2013 & Summer Internship, Caribbean Computing Excellence Institute, Caguas, PR. \\
\end{tabular}

\end{rSection}

%----------------------------------------------------------------------------------------
%	EXAMPLE SECTION
%----------------------------------------------------------------------------------------

%	Leadership Positions
%----------------------------------------------------------------------------------------
\begin{rSection}{Selected Leadership Positions}

\begin{rSubsection}{Solutions for Enterprise-Wide Procurement (SEWP)}{2021}{High Performance Computing SME}{Source Evaluation Board Member}
Served as a Subject Matter Expert of the Source Evaluation Board for the SEWP federal procurement program. Worked in the successful evaluation of technical proposals and contract selection, including budgeting and technical analysis.
\end{rSubsection}

\begin{rSubsection}{Partnership for Research and Education in Materials}{2018-2020}{Lead}{IRG2 Computational Research Team}
\item Lead a team of five undergraduate students in the task of computational material science, software engineering, and data science research. Prepared reports, posters and research presentations for Board and NSF stakeholders, including out of state conference presentations. 
\end{rSubsection}

\begin{rSubsection}{Partnership for Research and Education in Materials Nanodays}{2017-2020}{Student Representative}{IRG2 Research Team}
\item Gathered and communicated the ideas of all research students to the Board of PI's. Organized 100+ people research talks, symposiums, and provided assistance to new students. Taught material science and computer science topics to K-12 students in rural areas across the island.
\end{rSubsection}

\begin{rSubsection}{Society for Advancement of Chicanos/Hispanics Native Americans (SACNAS)}{2018-2019}{Journalist}{UPRH Chapter}
\item Served as lead journalist of the program. Documented and published articles displaying research performed by local students, outreach activities, upcoming events, and several analysis articles emphasizing outreach impacts to STEM programs at the University. 
\end{rSubsection}

\end{rSection}

%	TECHNICAL STRENGTHS SECTION
%----------------------------------------------------------------------------------------
\begin{rSection}{Selected Organizations}
American Geophysical Union (AGU) \\
NASA Intelligent Systems for Data Analysis Technologies (ISDAT) \\
NASA Hispanic Advisory Committee for Employees (HACE) \\
Society for Advancement of Chicanos/Hispanics and Native Americans in Science (SACNAS) \\
Emeritus Member of UPRH Association of Mathematics and Computer Science (ASMACC) \\
Materials Research Society (MRS) \\
\end{rSection}

%----------------------------------------------------------------------------------------

\begin{rSection}{Selected Awards and Recognitions}
Employee Exceptional Contribution Award, NASA HQ (2021), Washington D.C \\
Special Act Award - Individual, NASA GSFC (2021), Greenbelt, MD \\
Robert H. Goddard Exceptional Achievement for Mission Support, NASA (2019), Greenbelt, MD \\
Microsoft's PR Best Research Award (2019), Humacao, PR \\
NASA John Mather Scholarship Awardee (2017), Greenbelt, MD \\
Humacao Citizenship and Core Values Award (2016), Humacao, PR \\
NASA Minority Undergraduate Research Education Program Scholarship (2015-2018),  Greenbelt,  MD \\
University of El Turabo Best Values and Academic Achievement Award (2015), Gurabo, PR \\
Brystol Myers Squib Excellence in Science and Math Scholarship (2015), San Juan, PR \\
Top 5 Research Presentations, Material Research Society Conference (2014), Boston, USA \\
\end{rSection}

%\begin{rSection}{Section Name}

%Section content\ldots

%\end{rSection}

%----------------------------------------------------------------------------------------

\end{document}

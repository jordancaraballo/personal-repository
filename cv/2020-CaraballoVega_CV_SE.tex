%%%%%%%%%%%%%%%%%%%%%%%%%%%%%%%%%%%%%%%%%
% Medium Length Professional CV
% LaTeX Template
% Version 2.0 (8/5/13)
%
% This template has been downloaded from:
% http://www.LaTeXTemplates.com
%
% Original author:
% Trey Hunner (http://www.treyhunner.com/)
%
% Important note:
% This template requires the resume.cls file to be in the same directory as the
% .tex file. The resume.cls file provides the resume style used for structuring the
% document.
%
%%%%%%%%%%%%%%%%%%%%%%%%%%%%%%%%%%%%%%%%%

%----------------------------------------------------------------------------------------
%	PACKAGES AND OTHER DOCUMENT CONFIGURATIONS
%----------------------------------------------------------------------------------------

\documentclass{resume} % Use the custom resume.cls style

\usepackage[left=0.75in,top=0.6in,right=0.75in,bottom=0.6in]{geometry} % Document margins

\name{Jordan A. Caraballo-Vega} % Your name
\address{786-368-1596 \\ jordan.a.caraballo-vega@nasa.gov} % Your phone number and email
\address{github.com/jordancaraballo \\ linkedin.com/in/jordancaraballovega} % Your phone number and email

\begin{document}

%----------------------------------------------------------------------------------------
%	EDUCATION SECTION
%----------------------------------------------------------------------------------------

\begin{rSection}{Education}

{\bf University of Puerto Rico at Humacao} \hfill {\em 2015-2020} \\ 
Bachelor of Science - Computational Mathematics \\
Major Computer Science, GPA: 3.91/4.0 \smallskip \\
\end{rSection}

%----------------------------------------------------------------------------------------
%	WORK EXPERIENCE SECTION
%----------------------------------------------------------------------------------------

\begin{rSection}{Professional Experience}

%------------------------------------------------
\begin{rSubsection}{Science Data Processing Branch, Goddard Space Flight Center}{Fall 2018 - Present}{Computer Engineer AST}{Greenbelt, Maryland}

Responsible for the development of software to support science research in the areas of Earth Observation, artificial intelligence,  and hardware-accelerated applications.  Some ongoing projects include: land cover land use change (LCLUC), object detection, and semantic segmentation of multi-spectral remote sensing imagery.  Responsible for the system administration of mission critical GitLab and DevSecOps architecture,  together with Enterprise ELK log analysis cluster.
\end{rSubsection}

%------------------------------------------------

\begin{rSubsection}{Partnership for Research and Education in Materials (PREM), NSF}{2013 - 2020}{Computational Research Assistant}{Humacao, PR}
\item Lead a team of five undergraduate students in the development of material science computational projects. Developed and implemented software to accelerate molecular dynamics simulations in HPC environments. Assisted and collaborated with ongoing research projects in the area of data mining, clustering, image processing, and machine learning. Studied biological and electrical systems applied to physical sensors by means of molecular dynamics simulations.
\end{rSubsection}

\begin{rSubsection}{NASA Center for Climate Simulation, Goddard Space Flight Center}{June 2016 - July 2018}{Computer Science Trainee}{Greenbelt, Maryland}
\item[] Intern: Summer 2016, Fall 2016, Summer 2017, Summer 2018
\item[] Contractor under ADNET LLC: Spring 2017, Fall 2017, Spring 2018
\item Responsible for the engineering of software and information systems to support a critical log analysis infrastructure for monitoring and anomaly detection of high-performance computing systems. Served as a Linux and Unix system administrator for security and resource intensive systems, including revision control, networking, and user-facing virtual machines. Designed and developed software to support several computing intensive applications via multi-node MPI implementations and DevSecOps initiatives.
\end{rSubsection}

\end{rSection}

%----------------------------------------------------------------------------------------
%	TECHNICAL STRENGTHS SECTION
%----------------------------------------------------------------------------------------

\begin{rSection}{Technical Strengths}

\begin{tabular}{ @{} >{\bfseries}l @{\hspace{6ex}} l }
Programming Languages & Python, C++, Perl, NodeJS, Bash \\
Computing Skills & Data Mining, Image Processing, GPU, MPI \\
Machine/Deep Learning &  TensorFlow, PyTorch, NVIDIA RAPIDS, SkLearn \\
Remote Sensing & QGIS, GDAL, Rasterio \\
Simulation Skills & NAMD2, LAMMPS, GROMACS \\
Security Skills & ELK, Risk Assessment, Compliance and Vulnerabilities, Networking \\
DevOps & Containers, DevSecOps, Continuous Integration, Kubernetes \\
Operating Systems & System Administration of Linux, Unix, WindowsR2, FreeBSD \\
Website Development & HTML5, CSS, Javascript, Drupal \\
Soft Skills & Advanced Spanish, Advanced English, Effective Conflict Management \\
Security Clearance & Public Trust

\end{tabular}

\end{rSection}
%----------------------------------------------------------------------------------------
%	Publications
%----------------------------------------------------------------------------------------

\begin{rSection}{Peer Reviewed Publications}
\textbf{Abstract} - Cantu, L; Caraballo-Vega, J (2021) Application of the Data Science Workflow to Molecular Dynamic Simulations, Published Poster. \\
\textbf{Abstract} - Caraballo-Vega, J (2018) Millions of Messages per Minute! Surviving the NCCS's Log Avalanche, Published Demo. \\
\textbf{Abstract} - Caraballo-Vega, J (2017) Cybersecurity Machine Learning, Published Demo. \\
\textbf{Abstract} - Caraballo-Vega, J (2016) Building Cost Effective High Performance 100 Gbps Firewall, Published Demo. \\
\textbf{Article} - JAC, Mir., F.M. (2014) Molecular dynamics simulation of electrodes for capacitors made with nano-onions, NCUR. \\
\end{rSection}


%----------------------------------------------------------------------------------------
%	Most Recent Presentations
%----------------------------------------------------------------------------------------

\begin{rSection}{Most Recent Presentations}
Caraballo-Vega, J (2020) Deep Learning Techniques for the Classification of VHR Resolution Satellite Imagery, PREP-I, NASA Goddard Space Flight Center, Greenbelt, MD. \\
Caraballo-Vega, J (2020) Machine Learning Techniques for Protein-Type Classifications, NSF Science Symposium, Humacao, PR. \\
Caraballo-Vega, J (2020) Para-aminobenzamidine Spacer Arm Morphology Classification, SACNAS Conference, Honolulu, Hawaii. \\
Caraballo-Vega, J (2019) Computational Study of Gallium Crystals by Means of Molecular Dynamic Simulations, JTM Symposium, Caguas, PR. \\
Caraballo-Vega, J (2019) DevSecOps: Continuous Integration meets Containers Security, XXXVI Interdisciplinary Computer Science and Math Conference, Humacao, PR. \\
Caraballo-Vega, J (2018) Millions of Messages per Minute! Surviving the NCCS's Log Avalanche, 2018 Super Computing Conference, Denver, Colorado. \\
Caraballo-Vega, J (2018) Using Machine Learning for Anomaly Detection in HPC Environments, 2018 Institute for Teaching and Mentoring - South Reg. Education Board, Crystal City, VA.
\end{rSection}

%----------------------------------------------------------------------------------------
%	TECHNICAL STRENGTHS SECTION
%----------------------------------------------------------------------------------------

\begin{rSection}{Internships}

\begin{tabular}{ @{} >{\bfseries}l @{\hspace{6ex}} l }

June-August 2018 & Summer Internship, NASA Goddard Space Flight Center, Greenbelt, MD. \\
June-August 2017 & Summer Internship, NASA Goddard Space Flight Center, Greenbelt, MD. \\
August-December 2016 & Fall Internship, NASA Goddard Space Flight Center, Greenbelt, MD. \\
June-August 2016 & Summer Internship, NASA Goddard Space Flight Center, Greenbelt, MD. \\
June-August 2014 & REU, University of Pennsylvania, Philadelphia, PA. \\
June-August 2013 & Summer Internship, Caribbean Computing Excellence Institute, Caguas, PR. \\
\end{tabular}

\end{rSection}

%----------------------------------------------------------------------------------------
%	EXAMPLE SECTION
%----------------------------------------------------------------------------------------

%	Leadership Positions
%----------------------------------------------------------------------------------------
\begin{rSection}{Leadership Positions}

\begin{rSubsection}{Solutions for Enterprise-Wide Procurement (SEWP)}{2021-Present}{IT SME}{Source Evaluation Board Member}
Member of the Source Evaluation Board (SME) for the SEWP federal procurement program. Serve as the IT SME in the development of RFPs and contractual evaluations.
\end{rSubsection}

\begin{rSubsection}{Partnership for Research and Education in Materials}{2018-2020}{Lead}{IRG2 Computational Research Team}
\item Lead a team of five undergraduate students in the task of computational material science, software engineering, and data science research. Prepared reports, posters and research presentations for Board and NSF stakeholders. Performed laboratory technician work in the area of high-performance computing and networking.
\end{rSubsection}

\begin{rSubsection}{Partnership for Research and Education in Materials Nanodays}{2017-2020}{Student Representative}{IRG2 Research Team}
\item Gathered and communicated the ideas of all research students to the Board of PI's. Organized research talks, symposiums, and provided assistance to new students. Taught material science and computer science topics to K-12 students.
\end{rSubsection}

\begin{rSubsection}{Society for Advancement of Chicanos/Hispanics Native Americans (SACNAS)}{2018-2019}{Journalist}{UPRH Chapter}
\item Served as lead journalist of the program. Documented and published articles displaying research performed by local students, outreach activities, upcoming events, and several analysis articles emphasizing outreach impacts to STEM programs at the University.
\end{rSubsection}

\end{rSection}

%	TECHNICAL STRENGTHS SECTION
%----------------------------------------------------------------------------------------
\begin{rSection}{Organizations}
NASA Intelligent Systems for Data Analysis Technologies (ISDAT) \\
NASA Hispanic Advisory Committee for Employees (HACE) \\
Emeritus Member of UPRH Association of Mathematics and Computer Science (ASMACC) \\
Materials Research Society (MRS) \\
\end{rSection}

%----------------------------------------------------------------------------------------

\begin{rSection}{Awards and Recognitions}
Robert H. Goddard Exceptional Achievement for Mission Support (2019), Greenbelt, MD \\
Microsoft's PR Best Research Award (2019), Humacao, PR \\
John Mather Scholarship Awardee (2017), Greenbelt, MD \\
Humacao Citizenship Award (2016), Humacao, PR \\
NASA MUREP Scholarship Awardee (2015), Washington D.C \\
University of El Turabo Best Values and Academic Achievement (2015), Gurabo, PR \\
Brystol Myers Squib Excellence in Science and Math Scholarship (2015), San Juan, PR \\
Top 5 Research Presentations, Material Research Society Conference (2014), Boston, USA \\
\end{rSection}

%\begin{rSection}{Section Name}

%Section content\ldots

%\end{rSection}

%----------------------------------------------------------------------------------------

\end{document}
